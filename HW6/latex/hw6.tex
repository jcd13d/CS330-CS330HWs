\documentclass[letterpaper,11pt]{article}

\usepackage{fullpage,amsmath,amsfonts,latexsym,xcolor,clrscode3e}
\usepackage{graphicx}
\usepackage{amsthm}
\usepackage{hyperref}
\usepackage{fullpage}
\usepackage[ruled,vlined,linesnumbered]{algorithm2e}

\newcommand{\re}{{\mathbb{R}}}
\newcommand\numberthis{\addtocounter{equation}{1}\tag{\theequation}}
\newcommand{\floor}[1]{\lfloor {#1} \rfloor}
\newcommand{\ceil}[1]{\lceil {#1} \rceil}
\newcommand{\paren}[1]{\left( {#1} \right)}
\newenvironment{solution}{\color{black} }{}

\newcommand{\nats}{\mathbb{N}}

\newcommand{\comment}[1]{$\rhd$\ {\small\sf #1}}

\newtheorem{theorem}{Theorem}
\newtheorem{claim}[theorem]{Claim}
\newtheorem{lemma}[theorem]{Lemma}
\newtheorem{problem}{Problem}


\begin{document}
{\noindent\large
{\em Introduction to Analysis of Algorithms} \hfill \today\\
Boston University \hfill CS 330\\
Professor  Adam Smith, Dora Erdos \hfill Fall 2020\\}
\vspace{1pt}
\hrulefill\vspace{3mm}
\begin{center}
{\LARGE\bf Homework 6}\\
{\bf Due Wednesday, October 14 at 11:59 PM}
\end{center}
\begin{center}
    \color{teal}
   Student: Justin DiEmmanuele \\
    Collaborators: Shilpen Patel, George Padavick, Matthew Gilgo
\end{center}


\section*{Homework Guidelines}

\paragraph{Collaboration policy} Collaboration on homework problems, with the exception of
programming assignments and reading quizzes, is permitted, but not encouraged.
If you
choose to collaborate on some problems, you are allowed to discuss
each problem with at most 5 other students currently enrolled in the
class.
Before working with others on a problem, you should think about it
yourself for at least 45 minutes. Finding answers to problems on the
Web or from other outside sources (these include anyone not enrolled
in the class) is strictly forbidden.

{\em You must write up each problem solution by yourself without
assistance, even if you collaborate with others to solve the
problem.} You must also identify your collaborators. If you did not
work with anyone, you should write "Collaborators: none." It is a
violation of this policy to submit a problem solution that you
cannot orally explain to an instructor or TA.

\paragraph{Solution guidelines} For problems that require you to provide an algorithm, you must give the following:
    \begin{enumerate}
\item  a precise description of the algorithm in English and, if helpful, pseudocode,
\item a proof of correctness,
\item an analysis of running time and space.
\end{enumerate}
You may use algorithms from class as subroutines. You may also use any facts that we proved in class.


You should be as clear and concise as possible in your write-up of
solutions. 

A simple, direct analysis is worth more points than a
convoluted one, both because it is simpler and less prone to error and
because it is easier to read and understand. Points might be
subtracted for illegible handwriting and for solutions that are too
long. Incorrect solutions will get from 0 to 30\% of the grade,
depending on how far they are from a working solution. Correct
solutions with possibly minor flaws will get 70 to 100\%, depending on
the flaws and clarity of the write up.

\newpage 
\section{Djikstra and negative edges}

Usually, Djikstra's shortest-path algorithm is not used on graphs with negative-weight edges because it may fail and give us an incorrect answer.  However, sometimes Djikstra's will give us the correct answer even if the graph has negative edges.  (See Lab 4, question 1.2 for examples.)

You are given graph $G$ with at least one negative edge, and a source $s$.  Write an algorithm that tests whether Djikstra's algorithm will give the correct shortest paths from $s$.  If it does, return the shortest paths.  If not, return `no.'  The time complexity should not be longer than that of Djiksta's algorithm itself, which is $\Theta(|E| + |V|\log|V|)$.

\emph{(Hint: First, use Djikstra's algorithm to come up with candidate paths.  Then, write an algorithm to verify whether they are in fact the shortest paths from $s$.)}

\begin{itemize}
    \color{teal}
    \item \textbf{A precise description of the algorithm in English and, if helpful, 
        pseudocode.}

        The main idea of the algorithm is to run Dijkstra's algorithm to 
        get the claimed cheapest path and then for each edge "check" that 
        the distance of the child node is greater than or equal to the weight
        of the edge added to the distance of the root node as calculated by 
        Dijkstra's.

        For a graph $G$ represented by an adjacency list where each value is a 
        tuple with both the adjacent node and the weight $(v, w)$ and source 
        node $s$ the algorithm is defined as follows. 

        \begin{algorithm}[H]
            \color{teal}
            \caption{DjikstraTest(G, s)} 
            \KwIn{$G$ A graph in adjacency list form, with weights included}
            \KwIn{$s$ source node}
            $d = Dijkstras(G, s)$ \;
            \For{$i = 0$ to $len(G)$}{
                \For{ $j=0$ to $len(G[i])$ }{
                    $parent = i$ \;
                    $child = G[i][j][0]$ \;
                    $weight = G[i][j][1]$ \;
                }
                \If {$d[parent] + weight < d[child]$ }{
                    \KwOut{"no"}
                }
            }
            \KwOut{$d$}
        \end{algorithm}

    \item \textbf{Proof of correctness}
        
        If we assume that Dijkstra's algorithm returns the shortest distance 
        to each node, then for each edge $(u_i, v_i)$ with weight $w_i$, 
        \[
            d[u_i] + w_i \ge d[v_i]
        .\] 
        Where $d[]$ is the distance returned for that node by Dijkstra's. This 
        must be true because $d[u_i] + w_i = d[v_i]$ is the equation that finds
        the length of the path to $v_i$ that passes through the shortest path
        to $u_i$. If the distance to $v_i$ through $u_i$ is smaller than the 
        returned path length $d[v_i]$ there is clear inconsistency in Dijkstra's 
        result as Dijkstra's should have returned this path. 
        
        Our claim is that the returned Dijkstra's distances $d$ are equivalent 
        to the true shortest distances to each node from the source $d'$ if we 
        check each node in the graph using the algorithm above and the 
        condition is satisfied.

        For our base case, level zero of the Dijkstra's tree, we check that the 
        distance from the source (zero) plus the edge weight to each adjacent node 
        is greater than or equal to the claimed shortest path. If this is satisfied
        we know that there is no inconsistency in level zero.

        We assume this holds up to level $N$. For level $N + 1$ we run the same
        test. If the test is satisfied for each node in level $N + 1 $ we know 
        Dijkstra's paths are correct. 

    \item \textbf{Time Complexity}
        
        The time complexity of the algorithm is $O\left( n + m\log{n} \right) $
        where $n$ is the number of nodes and m is the number of edges. 
        This is the run time for Dijkstra's which the algorithm utilizes. The 
        next significant portion is iterating through each edge to check 
        the path lengths but this operation is only $O\left( n + m \right) $ 
        since it can only iterate through all the nodes and edges. All other 
        steps in the algorithm are $O\left( 1 \right) $ time.

    \item \textbf{Space Complexity}

        The space complexity is that of the space required to maintain the adjacency
        list representation of the graph - $O\left( m + n \right) $.


\end{itemize}





\newpage
\section{Scheduling a software project}
You are in charge of a software project that has $n$ subprojects.  Each subproject $s_i$ takes $t_i$ time to complete.  Some subprojects have dependencies, meaning that some subprojects cannot be started until certain other subprojects have been completed.  Otherwise, any number of subprojects can be performed simultaneously.

Write an algorithm that finds the shortest possible completion time for the project, as well as a schedule with start and finish times for each $s_i$ that achieves the shortest total time.

Here is an example:

\begin{center}
\begin{tabular}{ |c |c |c |}
\hline
{\bf Subproject} & {\bf $t_i$} & {\bf Dependencies} \\ 
\hline
$s_1$ & 5 & $s_4, s_{10}$ \\ 
 \hline 
$s_2$ & 20 & None \\
 \hline
$s_3$ & 5 & $s_6$\\ 
 \hline  
$s_4$ & 40 & None \\ 
 \hline 
$s_5$ & 15 & $s_9$ \\
 \hline
$s_6$ & 10 & $s_1, s_{10}$\\ 
 \hline  
$s_7$ & 15 & $s_2$ \\ 
 \hline 
$s_8$ & 35 & $s_2, s_4, s_7$ \\
 \hline
$s_9$ & 10 & None\\ 
 \hline  
$s_{10}$ & 30 & $s_2$ \\ 
 \hline 
\end{tabular}
\quad
\begin{tabular}{ |c |c |}
\hline
\multicolumn{2}{|c|}{\bf Schedule}\\
\hline
{\bf Subproject} & {\bf Start time}\\ 
\hline
$s_2$ & 0 \\ 
 \hline 
$s_4$ & 0  \\
 \hline
$s_9$ & 0 \\ 
 \hline  
$s_5$ & 10  \\ 
 \hline 
$s_7$ & 20  \\
 \hline
$s_{10}$ & 20 \\ 
 \hline  
$s_8$ & 40  \\ 
 \hline 
$s_1$ & 50  \\
 \hline
$s_6$ & 55 \\ 
 \hline  
$s_3$ & 65  \\ 
 \hline 
 \multicolumn{2}{|c|}{\bf Total time: 75}\\
 \hline
\end{tabular}
\end{center}


\emph{(Hint: How can you turn this problem into a graph so that you may use an algorithm you already know as part of the solution?)}

\newpage
\begin{itemize}

    \color{teal}
    \item \textbf{A precise description of the algorithm in English and, if helpful, pseudocode.}

        First, the provided node, dependency and weight values must be
        converted into a graph format. We will use an adjacency list. The 
        pseudocode to do so is provided below.

        \begin{algorithm}[H]
            \color{teal}
            \caption{Schedule(G)} 
            \KwIn{$JobTime$ array with time for each job $i$ }
            \KwIn{$Dependencies$ array of arrays containing the dependencies for each subproject $i$}
            $AdjList$ Initialize empty adjacency list \;
            \For{$i = 0$ to $len(JobTime)$}{
                \For{$j = 0$ to $len(Dependencies[i]$ }{
                    $AdjList[Dependencies[i][j]].append(\left( i, JobTime[i] \right) )$
                }
            }
            \KwOut{$AdjList$}
        \end{algorithm}

        The following algorithm takes an adjacency list representing the
        sub projects and their dependencies. The edges coming out of subproject
        $i$ will have weight $t_i$. To start, we must do a topological sort of
        the jobs. Then, we create an reverse adjacency list so we can look up 
        a nodes parents (dependencies) instead of it's children. 

        Finally, for each node in the order of the topological sort, we can 
        calculate start time $s_i$ by taking the maximum end time of each of 
        its parents. The end time $e_i$ of a node is $s_i + w_i$. The logic for 
        this is shown below.

        \begin{algorithm}[H]
            \color{teal}
            \caption{Schedule(G, t)} 
            \KwIn{$G$ A graph in adjacency list form, with weights included}
            \KwIn{$t$ array of subproject times for project $i$}
            $sorted = TopologicalSort(G)$ \;
            $G' = Reverse(G)$ \;
            $start = $ empty list default zero \;
            $end = $ empty list \;
            \For{$i = 0$ to $len(sorted)$}{
                $maxTime = 0$ \;
                $startTime = start[i]$ \;
                \For{$j = 0$ to $len(G'[i])$ }{
                    $weight = G'[i][j][1]$ \;
                    $newMax = weight + startTime$ \;
                    \If{$newMax > maxTime$ }{
                        $maxTime = newMax$
                    }
                }
                $start[i] = maxTime$ \;
                \For{$i = 0 $ to $len(t)$}{
                    $end[i] = start[i] + t[i]$ \;
                }
            }
            $totalTime = max(end)$\;
            \KwOut{$start, end, totalTime$}
        \end{algorithm}


    \item \textbf{Proof of Correctness}

        To prove correctness we need to prove the two following claims: The 
        longest path to each node gives the earliest possible start time and 
        iterating over the topological sort allows us to calculate these 
        longest paths.

        A job can only be run by definition if all of its prerequisites have 
        been met. We argue that finding the longest path of each node gives
        a time equivalent to the earliest time the job can be started.
        \begin{itemize}
            \item \textbf{Base Case}

                For the base case, we take the nodes with no dependencies. The 
                shortest path to node $v_i$ with no dependencies is obviously
                zero. This value is the expected minimum start time.

            \item \textbf{Induction Step}

                We assume that the longest path for each node $n$ steps from
                the root nodes represent the earliest time a job can be started.
                For node $n+1$ we know the earliest start time for each of its 
                parents.  The longest time from a parent node to this node is 
                the soonest time the job can be started since all of its 
                dependencies must be completed before start. The claim therefore
                holds for all $n$.

        \end{itemize}

        Next we prove that we can find this longest path on a topologically 
        sorted set. 
         \begin{itemize}
             \item \textbf{Base Case}

                 The base case is the first node in the topologically sorted
                 set. Since topological sort grantees there will be no 
                 dependencies to the left of any node in the array, we know 
                 the first node can take the default start time of zero. The 
                 algorithm returns the correct value. 

             \item \textbf{Induction Step}

                 We assume the algorithm works for nodes $0\ldots, n$. For case
                 $n + 1$ we must look at all dependencies earliest start time
                 and add the respective edge weights to find the longest path
                 to node $n + 1`$. This will only fail if we do not have a longest 
                 path for a parent of $n + 1$. Since the nodes are topologically 
                 sorted and being processed in that order, it is impossible that 
                 the algorithm will not have earliest starts for parent nodes
                 at level $n + 1$. Therefore the algorithm will give the desired
                 result.
         \end{itemize}

         Once we know we can find the start times for each node, finding the 
         end time is as simple as adding the provided job runtime. The total 
         time to finish all the jobs is just the maximum end time of the set 
         of jobs.

     \item \textbf{Time and Space Complexity}
         \begin{itemize}
             \item Time

                 The time complexity of this algorithm is $O\left( m + n \right) $.
                 The two main portions of the algorithm that are time intensive 
                 are the topological sort, the reverse of the graph, and the
                 calculation of the earliest start time iterating over the 
                 topologically sorted array. All three of these portions require
                 iterating over all the nodes and edges of the graph giving 
                 a time complexity of $O\left( m + n \right) $. The final step 
                 of calculating the end time and total start time can be done
                 in $O\left( n \right) $ time. All other sections of the algorithm
                 are $O\left( 1 \right) $ time.
             \item Space

                 The space complexity of the algorithm is $O\left( m + n  \right) $ 
                 as this is the space required to keep an adjacency list of 
                 the graph.
         \end{itemize}


\end{itemize}


\end{document}
