\documentclass[letterpaper,11pt]{article}

\usepackage{fullpage,amsmath,amsfonts,latexsym,xcolor,clrscode3e}
\usepackage{graphicx}
\usepackage{amsthm}
\usepackage{hyperref}
\usepackage{fullpage}
\usepackage[ruled,vlined,linesnumbered]{algorithm2e}

\usepackage{amsmath}

\newcommand{\re}{{\mathbb{R}}}
\newcommand\numberthis{\addtocounter{equation}{1}\tag{\theequation}}
\newcommand{\floor}[1]{\lfloor {#1} \rfloor}
\newcommand{\ceil}[1]{\lceil {#1} \rceil}
\newcommand{\paren}[1]{\left( {#1} \right)}
\newenvironment{solution}{\color{black} }{}

\newcommand{\nats}{\mathbb{N}}

\newcommand{\comment}[1]{$\rhd$\ {\small\sf #1}}

\newtheorem{theorem}{Theorem}
\newtheorem{claim}[theorem]{Claim}
\newtheorem{lemma}[theorem]{Lemma}
\newtheorem{problem}{Problem}


\begin{document}
{\noindent\large
{\em Introduction to Analysis of Algorithms} \hfill \today\\
Boston University \hfill CS 330\\
Professor  Adam Smith, Dora Erdos \hfill Fall 2020\\}
\vspace{1pt}
\hrulefill\vspace{3mm}
\begin{center}
{\LARGE\bf Homework 8}\\
{\bf Due Wednesday, November 4 at 11:59 PM}
\end{center}
\section*{Homework Guidelines}

\paragraph{Collaboration policy} Collaboration on homework problems, with the exception of
programming assignments and reading quizzes, is permitted, but not encouraged.
If you
choose to collaborate on some problems, you are allowed to discuss
each problem with at most 5 other students currently enrolled in the
class.
Before working with others on a problem, you should think about it
yourself for at least 45 minutes. Finding answers to problems on the
Web or from other outside sources (these include anyone not enrolled
in the class) is strictly forbidden.

{\em You must write up each problem solution by yourself without
assistance, even if you collaborate with others to solve the
problem.} You must also identify your collaborators. If you did not
work with anyone, you should write "Collaborators: none." It is a
violation of this policy to submit a problem solution that you
cannot orally explain to an instructor or TA.

\paragraph{Solution guidelines} For problems that require you to provide an algorithm, you must give the following:
    \begin{enumerate}
\item  a precise description of the algorithm in English and, if helpful, pseudocode,
\item a proof of correctness,
\item an analysis of running time and space.
\end{enumerate}
You may use algorithms from class as subroutines. You may also use any facts that we proved in class.


You should be as clear and concise as possible in your write-up of
solutions. 

A simple, direct analysis is worth more points than a
convoluted one, both because it is simpler and less prone to error and
because it is easier to read and understand. Points might be
subtracted for illegible handwriting and for solutions that are too
long. Incorrect solutions will get from 0 to 30\% of the grade,
depending on how far they are from a working solution. Correct
solutions with possibly minor flaws will get 70 to 100\%, depending on
the flaws and clarity of the write up.

\newpage 
\begin{enumerate}
    \item \textbf{(Recurrences)}
    
    Consider the following algorithm $A$:
\begin{verbatim}
def A(n):
    if n == 0:
        return 0
    if n is even:
        return A(n/2)
    else:
        return A(2(n-1)) + 1
\end{verbatim}

\begin{enumerate}
    \item Prove that the $A$ terminates when its input $n$ is a nonnegative integer.
    \item Write a recurrence for the running time (in terms of $n$) of $A$.
    \item Give a closed form for its asymptotic running time (using whichever method you like)
    \item If we look at the tree of recursive calls to $A$ made by this algorithm, what is its depth (asymptotically, in terms of $n$)? How many leaves does it have (asymptotically, in terms of $n$)?
     \item What function of $n$ does $A$ compute? It's easy to express it in terms of the binary expansion of $n$ (i.e. the bits you get when you write $n$ in base 2).
    \end{enumerate}

    
    \begin{enumerate}
        \color{teal}
        \item To prove that $A$ terminates when its input of $n$ is a 
            non-negative integer, we will examine the three cases individually. 
            We hope to show that in each case the input $n$ can only decrease
            in size and eventually will hit the base case.
            \begin{itemize}
                \item Case 1: $n == 0$

                    If $n == 0$, the algorithm simply returns zero and will 
                    terminate. 

                \item Case 2: The input $n$ is even

                    If the input is even, the algorithm $A$ is called recursively
                    on $\frac{n}{2}$ so the input is cut in half.  

                \item Case 3: The input $n$ is odd

                    If the input is odd we see that $A$ returns 
                    $A(2\left( n - 1 \right)) + 1$. Here it is less clear
                    that $n$ should always decrease in a series of calls. 
                    However, we can continue from this case to see if we can 
                    discover anything about the subsequent calls.
                    
                    \begin{itemize}
                        \item First we recognize that for any odd number $n$, 
                            $n - 1$ is even. 
                        \item Since the next call will be on an even number,
                            we know that case 2 will be true. We can then 
                            substitute the original $2\left( n - 1 \right) $ 
                            into case 2.
                            \[
                                A(\frac{2\left( n - 1 \right)}{2}) = A(n-1)
                            .\] 
                        \item From the call above, we see that the passed value
                            resolves to $n-1$. Since we know $n-1$ is even,
                            the next call must be:
                             \[
                                 A\left( \frac{n-1}{2} \right) 
                            .\] 
                    \end{itemize}

                From these points we see that after 3 calls to $A$ consisting
                of constant-time operations we arrive at a point where the 
                input is once again halved. We finally must recognize that if 
                any even number is halved and any odd number is decremented by
                one and then halved we must eventually arrive at $n = 1$. This
                input would call case 3: 
                $A\left( 2\left( 1-1 \right)  \right) + 1$ in which the 
                recursive call reaches the base case of $n=0$.\\

                If $A$ is always taking constant time steps to half the input
                and it will eventually reach the base case, the algorithm must 
                terminate. 
            \end{itemize}
        \item Written recurrence:
            \[ 
                \begin{cases}
                    0 & x\leq 0 \\
                    \frac{100-x}{100} & 0\leq x\leq 100 \\
                    0 & 100\leq x
                \end{cases}
            \]
    \end{enumerate}


    \newpage
    \item \textbf{(More recurrences)}
    
    Suppose you are choosing between the following three algorithms, all of which have O(1) base cases for size 1:
\begin{enumerate}
        \item Algorithm A solves problems of size $n$ by dividing them into five subproblems of size $n/2$, recursively solving each subproblem, and then combining the solutions in linear time.
        \item Algorithm B solves problems of size $n$ by recursively solving one subproblem of size $n/2$, one subproblem of size $2n/3$, and one subproblem of size $3n/4$ and then combining the solutions in linear time. 
        
        \emph{Hint:}  Approach this algorithm via the substitution method (pp. 211-217) to avoid tough summations.  
        Also, solving that one as O($n^d$) for the smallest valid integer $d$ is all we're looking for.
        \item Algorithm C solves problems of size $n$ by dividing them into nine subproblems of size $n/3$, recursively solving each subproblem, and then combining the solutions in $O(n^{2})$ time.
\end{enumerate}
What are the running times of each of these algorithms (in asymptotic notation) and which would you choose?  You may use the
Master Method.

\end{enumerate}



\end{document}
