\documentclass[letterpaper,11pt]{article}

\usepackage{fullpage,amsmath,amsfonts,latexsym,xcolor,clrscode3e}
\usepackage{graphicx}
\usepackage{amsthm}
\usepackage{hyperref}
\usepackage{fullpage}
\usepackage[ruled,vlined,linesnumbered]{algorithm2e}

\newcommand{\re}{{\mathbb{R}}}
\newcommand\numberthis{\addtocounter{equation}{1}\tag{\theequation}}
\newcommand{\floor}[1]{\lfloor {#1} \rfloor}
\newcommand{\ceil}[1]{\lceil {#1} \rceil}
\newcommand{\paren}[1]{\left( {#1} \right)}
\newenvironment{solution}{\color{black} }{}

\newcommand{\nats}{\mathbb{N}}

\newcommand{\comment}[1]{$\rhd$\ {\small\sf #1}}

\newtheorem{theorem}{Theorem}
\newtheorem{claim}[theorem]{Claim}
\newtheorem{lemma}[theorem]{Lemma}
\newtheorem{problem}{Problem}


\begin{document}
{\noindent\large
{\em Introduction to Analysis of Algorithms} \hfill \today\\
Boston University \hfill CS 330\\
Professor  Adam Smith, Dora Erdos \hfill Fall 2020\\}
\vspace{1pt}
\hrulefill\vspace{3mm}
\begin{center}
{\LARGE\bf Homework 9}\\
{\bf Due Wednesday, November 11 at 11:59 PM}
\end{center}


\begin{center}
    \color{teal}
   Student: Justin DiEmmanuele \\
    Collaborators: Shilpen Patel, George Padavick, Matthew Gilgo
\end{center}


\section*{Homework Guidelines}

\paragraph{Collaboration policy} Collaboration on homework problems, with the exception of
programming assignments and reading quizzes, is permitted, but not encouraged.
If you
choose to collaborate on some problems, you are allowed to discuss
each problem with at most 5 other students currently enrolled in the
class.
Before working with others on a problem, you should think about it
yourself for at least 45 minutes. Finding answers to problems on the
Web or from other outside sources (these include anyone not enrolled
in the class) is strictly forbidden.

{\em You must write up each problem solution by yourself without
assistance, even if you collaborate with others to solve the
problem.} You must also identify your collaborators. If you did not
work with anyone, you should write "Collaborators: none." It is a
violation of this policy to submit a problem solution that you
cannot orally explain to an instructor or TA.

\paragraph{Solution guidelines} For problems that require you to provide an algorithm, you must give the following:
    \begin{enumerate}
\item  a precise description of the algorithm in English and, if helpful, pseudocode,
\item a proof of correctness,
\item an analysis of running time and space.
\end{enumerate}
You may use algorithms from class as subroutines. You may also use any facts that we proved in class.


You should be as clear and concise as possible in your write-up of
solutions. 

A simple, direct analysis is worth more points than a
convoluted one, both because it is simpler and less prone to error and
because it is easier to read and understand. Points might be
subtracted for illegible handwriting and for solutions that are too
long. Incorrect solutions will get from 0 to 30\% of the grade,
depending on how far they are from a working solution. Correct
solutions with possibly minor flaws will get 70 to 100\%, depending on
the flaws and clarity of the write up.

\newpage 
\begin{enumerate}
\item (\textbf{Recursion vs. Memoization})

You are playing a board game in which you move your pawn along a path of  $n$ fields. Each field has a number $\ell$ on it. In each move, if your field carries the number $\ell$, then you can choose to take any number of steps between  $1$ and $\ell$. You can only move forward, never back. Your goal is to reach the last field in as few moves as possible. \\
Below we've written a recursive algorithm to find the optimal strategy for a given
board layout. The input to the algorithms is an array \texttt{board}
containing the numerical value in each field. The first field of the
game corresponds to \texttt{board[0]} and the last to
\texttt{board[n-1]}. Here we use Python conventions for range so
\texttt{range(a,b)} consists of integers from \texttt{a} to
\texttt{b-1} inclusively.

\begin{verbatim}
recursive_moves(array board, int  L, int n ):
   # board has length n and contains only positive integers.
   if L == n-1:
     return 0
   min_so_far = float('inf')
   for i in range(L + 1, min(L + board[L] +1, n) ): 
      moves = recursive_moves (board, i, n) 
      if (moves + 1 < min_so_far): 
         min_so_far = moves + 1
   return min_so_far
\end{verbatim}

\begin{enumerate}\renewcommand{\theenumi}{1.\arabic{enumi}}
\item (1 pt.) Run \texttt{recursive\_moves(board,0,len(board))} with
  the input array $ [1, 3, 6, 3, 2, 3, 9, 5]$. Write down
  a list of all the distinct calls to the function (making sure to note the value of
  input \texttt{L}) and the return values. It's ok to ``memoize'' as
  you go. 

  \begin{itemize}
      \color{teal}
      \item The table below contains all of the distinct calls to the function
          and their return values. 

          \begin{table}[htpb]
              \color{teal}
              \centering
              \caption{Distinct Calls}
              \label{tab:DistinctCalls}
              \begin{tabular}{c | c}
              Call  & Return Value \\  
              \hline
              recursive\_moves(board, 0, n) & 3 \\ 
              recursive\_moves(board, 1, n) & 2 \\ 
              recursive\_moves(board, 2, n) & 1 \\ 
              recursive\_moves(board, 3, n) & 2 \\ 
              recursive\_moves(board, 4, n) & 2 \\ 
              recursive\_moves(board, 5, n) & 1 \\ 
              recursive\_moves(board, 6, n) & 1 \\ 
              recursive\_moves(board, 7, n) & 0 \\ 
              \end{tabular}
          \end{table}
  \end{itemize}
  

  \item (2 pts) Show that the algorithm is correct. It will help to formulate
    a claim of the form: ``On input \texttt{L}, the algorithm correctly finds
    the number of moves needed to get from position $X$ to position
    $Y$ of the board'' (for some values of $X$ and $Y$ that depend on \texttt{L}).
    \begin{itemize}
        \color{teal}
    \item Claim: The algorithm finds the correct value for all 
        $L  \in \{0, \ldots, n-1\} $ for board with $n$ spaces.
    \item Base Case: $L = n - 1$. When $L = n - 1$ the algorithm triggers the 
        first if statement and returns the expected zero steps to the end
        of the board since we are already at the end. 
    \item Assume: The algorithm is correct for all  
        $L \in \{L+1, \ldots, L+board[L]\} $.
    \item We note that all the sub problems  $L^{\prime}$ where 
    $L^{\prime} > L$ are smaller than $L$ up to the smallest problem of the
        base case where the algorithm simply returns zero. This proves that 
        the algorithm will terminate as it only progressively calls smaller 
        sub problems.
    \item We seek to prove correctness for position $L$.\\

        The algorithm makes calls to $\{L + 1, \ldots, L+board[L]\} $ which return
        correctly per the inductive hypothesis. We are interested in taking 
        whichever number of steps that minimizes the total number of steps 
        to the end of the board. To do this we want to use our one step from
        position $L$ get to the minimum reachable cell which are covered 
        by the induction hypothesis. The algorithm indeed takes the minimum 
        of these possible steps and returns it with the addition of 1
        for the current step correctly.
    \end{itemize}
    


\item (1 pts.) Show that the worst-case running time of
  \texttt{recursive\_moves(board,0,len(board))} (as written, with no
  memoization) on an input array of
  size $n$ is $\Omega(2^n)$.
  \begin{itemize}
      \color{teal}
  \item In the worst-case, the board will have values $board[i] > n - i$ where 
      $i \in \{0, \ldots, n-1\} $ and $n$ is the length of the board because at
      each iteration the algorithm will have to look for the minimum of the max
      number of "min\_so\_far" values. 
  \item To find how many calls are needed in this worst-case condition, we can 
      follow the number of calls at each step from position  $n - 1$ to $0$. 
      Again note that at each position $i$ the function must be called for 
      $i$ to $n - 1$.
      \begin{itemize}
          \item Position $n - 1 $: Base case terminates after \textbf{1 call}. 
          \item Position $n - 2$: Must be called on the base (1 call) and 
              itself (1 call) \textbf{2 calls}.
          \item Position $n - 3$: The algorithm must be called on itself, $n - 2$ and $n - 1$. This gives  $1 + 2 + 1 = 4$ \textbf{4 calls}.
          \item Position $n - 4$: Must call $n - 3$, $n - 2$ and $n - 1$ in 
              addition to itself. This gives $1 + 4 + 2 + 1 = 8$ 
              \textbf{8 calls}.
      \end{itemize}
  \item The pattern begins to become clear here: as the size increases by 1
      the number of calls for that next item doubles the number of calls
      for the entire input $n-1$. This is consistent with a run time of 
      $\Omega\left(  2^{n}\right) $.
  \end{itemize}
  


  \item (1 pts) For how many \emph{distinct} values of the inputs will
    the algorithm make recursive calls on
    inputs of size $n$ in the worst case?
    \begin{itemize}
        \color{teal}
        \item For an input of size $n$ there will be at most $n$ distinct 
            values of the inputs to the algorithm. 
    \end{itemize}
    
    \item (3 pts.) Write  pseudocode (or working code) for a memoized version of the \texttt{recursive\_moves} procedure that runs in polynomial time.

    \begin{itemize}
        \color{teal}
        \item Below is the memoized version of the algorithm.

\begin{verbatim}
def recursive_moves(array board, int  L, int n, array memo):
    if memo is None:
        memo = [None]*n
    if memo[L] is not None:
        return memo[L]
    if L == n-1:
        return 0
    min_so_far = float('inf')
    for i in range(L + 1, min(L + board[L] +1, n) ): 
        moves = recursive_moves (board, i, n, memo) 
        if (moves + 1 < min_so_far): 
            min_so_far = moves + 1
    memo[L] = min_so_far
    return min_so_far
\end{verbatim}

        \item The array "memo" is added in this version and initialized to
            being null. If the algorithm is called, it first checks if memo 
            contains a solution to the current sub problem. If there is no 
            saved solution, the algorithm continues to calculate it and it is
            saved into memo as the last step of the algorithm. This guarantees 
            that a sub problem will only be calculated once. Any call to a 
            previously calculated problem will run in $O\left( 1 \right) $ time.

    \end{itemize}

    \item (3 pts.) Write  pseudocode (or working code)  for a nonrecursive  ``bottom-up'' version of the \texttt{recursive\_moves} procedure that runs in polynomial time. Note that ``bottom-up'' here does not necessarily mean in increasing order of \texttt{L}; it means from the simplest subproblems to the most complicated ones.[\emph{Hint}: You might want to program your algorithms yourself to
  test them. They should be short and easy to code.]

  \begin{itemize}
      \color{teal}
      \item Below is a "bottom-up" version of the algorithm.
\begin{verbatim}
def recursive_moves(board, L, n):
    OPT = [math.inf]*n
    for i in range(n - 1, -1, -1):
        if i == n - 1:
            OPT[i] = 0
        else:
            print(OPT)
            OPT[i] = min(OPT[i: min(i + board[i] + 1, n)]) + 1
    return OPT[L]

\end{verbatim}
    
    \item The above bottom up version starts with the smallest sub problem
        at position  $n-1$. Just the base case in previous algorithms this 
        position is set to the correct number of zero steps since we would
        already be at the end of the board. 
    \item The algorithm then iterates from $n-1$ down to position zero of the 
        board calculating the optimum number of steps for each position along
        the way. The solution at each step is only dependent on the positions 
        greater than itself so we can guarantee the algorithm will have the 
        information needed to calculate $OPT[i]$.
        \begin{itemize}
            \item To prove needed solution values will be available we can 
                assume that the information at step $L + i$ is not available.
                If $OPT[L+i]$ is not available, we must not have iterated 
                through position $L + i$ yet - but since the for loop decrements
                from $n-1$ to $0$, it is impossible that $L$ has been reached 
                without initializing $OPT[L+i]$ for all $i$.
        \end{itemize}
    \item The algorithm uses the same principle as the original algorithm - 
        for any position $L$ the min number of steps to the end of the board
        is $min \{OPT(L + 1), \ldots,  OPT(L + board[L])\} + 1$. In this 
        algorithm the step $L$ is included in the min statement but it is 
        initialized to infinity so it will never be chosen. 



  \end{itemize}


\item (1 pts.) Analyze the asymptotic worst-case running time of \emph{your} algorithms on input
  arrays of size $n$. (The two algorithms will probably be the same).
  \begin{itemize}
      \color{teal}
      \item Both algorithms will consider each index in the array to find 
          the optimum solution at that position. This first does this 
          recursively and the second does it iteratively. This takes 
          $O\left( n \right) $ time.
      \item During the call at each position the algorithm must decide which
          position to take a step to to minimize the overall number of steps. 
          The algorithm will consider $n - L$ positions at each step in the 
          worst-case. This takes $O\left( n \right) $ time.
      \item This makes the overall run time in both cases 
          $O\left( n^{2} \right) $.
  \end{itemize}
 
\end{enumerate}
\newpage

\item \textbf{(BST key interval)}

Your friend is given the following task: Given the root of a BST with depth $h$ and an interval $[a,b]$ as input, write an algorithm that prints all keys in the interval in time $O(h+k)$ where $k$ is the number of keys matching the search. (Assume $a<b$).

Your friend is unsure of whether their code works, so they ask you to take a look at it:

\begin{verbatim}
def find_keys(root, a, b):
1    if (root == NULL):
2        return
3    if (a < root.key):  
4        find_keys(root.left, a, b)
5    else if (a <= root.key AND b >= root.key):  
6        print(root.key)
7    if (b > root.key):  
8        find_keys(root.right, a, b) 
9    return
\end{verbatim}


You decide that the code is not correct.
\begin{enumerate}
    \item Suggest a \textbf{single line} to change which would make the algorithm correct (you may delete what your friend has on that line). List the line number and the exact code that you would put on that line (please use the same syntax and variable names as the pseudocode e.g. \texttt{root.left, root.right, root.key, a, b}). Hint: the error is not syntactic or something else trivial.

        \begin{itemize}
            \color{teal}
            \item Line 5 should be changed to the following:
\begin{verbatim}
    if (a <= root.key AND b >= root.key):
\end{verbatim}
        \end{itemize}

    \item Prove the modified algorithm's correctness (with your single line change).

        \begin{itemize}
            \color{teal}
            \item Claim: the algorithm "find\_keys" will print all keys in the
                interval $[a, b]$.
            \item Base Case: node passed is null - the algorithm correctly 
                will not print the key of this node. 
            \item Assume: The algorithm correctly prints the keys of values 
                within the interval $[a, b]$ for nodes in levels $root + 1$ to 
                $h$.
            \item We will prove correctness at $root$ level to complete the 
                induction step. There are three cases to consider.
                \begin{itemize}
                    \item Case 1: The key of root satisfies 
                        $a \le root.key \le b$. In this 
                        case, the algorithm is called on values less than
                        $root$ by recursively calling on $root.left$ which will
                        correctly print any value $n$ where $a \le n \le root$ 
                        per the induction hypothesis (assumption). The 
                        algorithm will then correctly print $root.key$ 
                        as the node falls within the interval. 
                        Finally, the algorithm will call
                        recursively on values greater than $root.key$ by 
                        passing $root.right$. This is correct per the induction
                        hypothesis. 
                    \item Case 2: The key of root satisfies $a > root.key$. In
                        this case the only code triggered is a recursive call
                        on  $root.right$ or greater values than $root.key$. 
                        This is correct by they induction hypothesis and will
                        move the recursive calls in the correct direction of
                        higher values. 
                    \item Case 3: the key of root satisfies $b < root.key$. In 
                        this case the only code triggered is a recursive call
                        on $root.left$. This is correct by induction hypothesis
                        and moves the recursive call in the correct direction
                        of lower values. 
                \end{itemize}
        \end{itemize}

    \item (Choose one) Which of the following runtimes is the \emph{lowest} upper bound on the runtime of your modified algorithm: ($h$ is the height of the tree, $k$ is the number of keys matching the search, and $n$ is the number of nodes in the tree). No need to justify the answer.
    \begin{enumerate}
        \item $O(n)$
        \item $O(h+k)$
        \item $O(h\log k)$
        \item $O(k^2)$
    \end{enumerate}
    
    [Hint to analyze running time: first give an upper bound on the number of nodes that this algorithm will visit \textit{other} than the ones with keys in $[a,b]$.]
    \begin{itemize}
        \color{teal}
        \item \textbf{ii}
    \end{itemize}
\end{enumerate}


    

\end{enumerate}



\end{document}
