\documentclass[letterpaper,11pt]{article}

\usepackage{fullpage,amsmath,amsfonts,latexsym,xcolor,clrscode3e}
\usepackage{graphicx}
\usepackage{amsthm}
\usepackage{hyperref}
\usepackage{fullpage}
\usepackage[ruled,vlined,linesnumbered]{algorithm2e}

\newcommand{\re}{{\mathbb{R}}}
\newcommand\numberthis{\addtocounter{equation}{1}\tag{\theequation}}
\newcommand{\floor}[1]{\lfloor {#1} \rfloor}
\newcommand{\ceil}[1]{\lceil {#1} \rceil}
\newcommand{\paren}[1]{\left( {#1} \right)}
\newenvironment{solution}{\color{black} }{}

\newcommand{\nats}{\mathbb{N}}

\newcommand{\comment}[1]{$\rhd$\ {\small\sf #1}}

\newtheorem{theorem}{Theorem}
\newtheorem{claim}[theorem]{Claim}
\newtheorem{lemma}[theorem]{Lemma}
\newtheorem{problem}{Problem}


\begin{document}
{\noindent\large
{\em Introduction to Analysis of Algorithms} \hfill \today\\
Boston University \hfill CS 330\\
Professor  Adam Smith, Dora Erdos \hfill Fall 2020\\}
\vspace{1pt}
\hrulefill\vspace{3mm}
\begin{center}
{\LARGE\bf Homework 7}\\
{\bf Early deadline: Wednesday, October 21 at 11:59 PM \\ No penalty later deadline: Saturday, October 24 at 11:59 PM}
\end{center}

\begin{center}
    \color{teal}
   Student: Justin DiEmmanuele \\
    Collaborators: Shilpen Patel, George Padavick, Matthew Gilgo
\end{center}

\section*{Homework Guidelines}

\paragraph{Collaboration policy} Collaboration on homework problems, with the exception of
programming assignments and reading quizzes, is permitted, but not encouraged.
If you
choose to collaborate on some problems, you are allowed to discuss
each problem with at most 5 other students currently enrolled in the
class.
Before working with others on a problem, you should think about it
yourself for at least 45 minutes. Finding answers to problems on the
Web or from other outside sources (these include anyone not enrolled
in the class) is strictly forbidden.

{\em You must write up each problem solution by yourself without
assistance, even if you collaborate with others to solve the
problem.} You must also identify your collaborators. If you did not
work with anyone, you should write "Collaborators: none." It is a
violation of this policy to submit a problem solution that you
cannot orally explain to an instructor or TA.

\paragraph{Solution guidelines} For problems that require you to provide an algorithm, you must give the following:
    \begin{enumerate}
\item  a precise description of the algorithm in English and, if helpful, pseudocode,
\item a proof of correctness,
\item an analysis of running time and space.
\end{enumerate}
You may use algorithms from class as subroutines. You may also use any facts that we proved in class.


You should be as clear and concise as possible in your write-up of
solutions. 

A simple, direct analysis is worth more points than a
convoluted one, both because it is simpler and less prone to error and
because it is easier to read and understand. Points might be
subtracted for illegible handwriting and for solutions that are too
long. Incorrect solutions will get from 0 to 30\% of the grade,
depending on how far they are from a working solution. Correct
solutions with possibly minor flaws will get 70 to 100\%, depending on
the flaws and clarity of the write up.

\paragraph{Exercises (do not hand in)}
The following exercises are meant for practicing using the cut and cycle properties.

\begin{itemize}
    \item Given an edge $e$ in the MST of a graph $G$, find and output a cut $S$ for which it is the lightest edge in the cutset.
    \item Given an edge $e$ not in the MST, find and output a cycle $C$ on which it is the heaviest edge.
\end{itemize}

\section*{Homework problems to hand in}
\newpage
\begin{enumerate}
    \item (\textbf{MST Updates}) Suppose you are given a graph $G$ (with unique edge weights) and its associated minimum spanning tree $M$, along with an edge $(u,v)\in G$ that we want to remove. Call the resulting graph with the edge removed $G'$. Assume that $G'$ is still connected. 

    \begin{enumerate}
        \item Argue that it's sufficient to change at most one edge in the MST for $G$ so that it's the MST for $G'$. How do the cycle and cut properties help you find the edge to change? \textit{Hint:} Think about two cases separately: $(u,v) \in M$ and $(u,v) \notin M$.

        \item Write an algorithm that takes $G$, the MST of $G$, and an edge $e$ in $G$, and outputs the minimum spanning tree of $G'$ (again, assuming $G'$ is still connected). Your algorithm should run in $O(n+m)$ time (do not recompute the MST from scratch; instead modify the MST you are given).
    \end{enumerate}
    
    Below are some examples of graphs with MST edges labeled, and the resulting $G'$ and MST of $G'$
    
    \begin{enumerate}
        \color{teal}
        \item \textbf{It is sufficient to change at most one edge in the MST}
            \begin{itemize}
                \item Case 1: $(u, v)\not\in M$

                    We claim that if the deleted edge $(u, v)$ is not in the MST $M$
                we do not need to change $M$. For any edge $e$ in $M$, we can 
                remove $e$ to create two separate trees, neither of which are 
                spanning any longer. Let us call the two sets of nodes created
                $A$ and $B$. By removing any edge $e$ and taking the cut set 
                $(A, B)$, we can use the cut property outlined in the textbook
                to prove that the minimum weight edge in that set must be included
                in the new MST $M^{\prime}$. Since the previous graph was an
                MST and the only deleted edge is one that was not in $M$ 
                $M^{\prime}$ must be equivalent to $M$ or else the incoming 
                tree $M$ must not be an MST.

            \item Case 2: $(u, v) \in M$

                The argument from case 1 still applies and shows that any edge
                in $M$ that has not been deleted should be kept in $M^{\prime}$.
                Now, we must decide which edge should take the deleted edge's 
                place. Since we are guaranteed in the problem statement that the
                graph will still be connected after the edge is deleted we know 
                there will be at least one candidate to connect the disconnected
                trees $A$ and $B$ created by deleting $e \in M$. We can use the 
                cut property on the cut set $(A, B)$ to add the lightest edge
                that connects those two sub-graphs. By the cut property we know
                that edge must be in $M^{\prime}$.

            \end{itemize}

        \item \textbf{Write an algorithm to find the new MST}
            \begin{itemize}
                \item \textbf{A precise description of the algorithm}

                \begin{algorithm}[H]
                    \color{teal}
                    \caption{NewMST($G$, $M$, $(u, v)$, $W$)} 
                    \KwIn{$G$ A graph in adjacency list form - tuples of (adj\_node, weight)}
                    \KwIn{$M$ Minimum spanning tree of graph $G$ in adjacency list form}
                    \KwIn{$(u, w)$ edge to be deleted in  $G$}
                    \KwIn{$W$ Weight of edge $i$}
                    $G^{\prime}$ = $G$ \;
                    $G^{\prime}[u][v] = 0$\;
                    $min\_weight =$ inf\;

                    \If{$M[u][v] = 0$ }{
                        \KwOut{$M$}
                    }

                    $group_1$ = BFS($G^{\prime}$, $u$ )\;
                    $group_2$ = BFS($G^{\prime}$, $v$ )\;

                    \For{$i=0$ to $len(G) - 1$}{
                        \For{$j=0$ to $len(G[i]) - 1$ }{
                            \If{($group_1[u]$  not null and $group_2[v]$ not null) 
                                or ($group_1[v]$  not null and $group_2[u]$ not null)}{
                                \If{$G^{\prime}[i][j][1] < min\_weight$}{
                                    $min\_weight = G^{\prime}[i][j][1]$\;
                                    $new\_edge = (i, j)$\;
                                }
                            }
                        }
                    }

                    $M^{\prime} = M$ \;
                    $M^{\prime}[u][v] = 0$ \;
                    $M^{\prime}[new\_edge[0]][new\_edge[1] = 1$ \;

                    \KwOut{$M^{\prime}$}
                \end{algorithm}

                The algorithm first checks in line 4 if the edge exists in the 
                MST - if not it returns the old MST. It then creates a record
                of the nodes in the two disconnected trees created if the 
                deleted node is in $M$. For each edge we then check if one node
                exists in group 1 while the other node exists in group 2. If this
                is true, we check the edge weight and update our minimum weight
                edge if this edge weight is lower than the known lowest edge weight.
                Finally, we add the new lowest weight edge that connects groups 
                1 and 2 to the new MST $M^{\prime}$ and return it.\\


                \item \textbf{Proof of correctness}\\
                     
                    We have already proved that any edge in $M$ after deletion 
                    will be in $M^{\prime}$ and that if the deleted edge is in
                    $M$, we must simply take the lightest edge in the cut set 
                    between the two remaining tees.\\

                    This is exactly what the algorithm sets out to do, we must 
                    only prove that it achieves that goal. First, we initialize 
                    variables and set the new graph $G^{\prime}$ this is 
                    straightforward. Next, we find the disconnected trees in 
                    group 1 and 2. Since an MST is a tree, definitionally, 
                    deleting an edge in it will create two disconnected trees. 
                    We know that BFS will return the lengths to any node in a
                    connected component so we use that to find which nodes exist
                    in which sub tree of the original MST. \\

                    We then iterate through all edges and only take a new lowest
                    weight if the edge has one node in each of the disconnected 
                    trees. Since this iterates through every edge we are guaranteed 
                    to find the edge of lowest weight that connects the groups.\\

                    Finally, we add that edge to $M ^{\prime}$ and return it to the 
                    user. Each step of the algorithm does what is needed to reach
                    the stated goal.\\

                \item \textbf{Time and space complexity}
                    
                    \textbf{Time Complexity}: Up until line 5 we only call 
                    operations that run in $O\left( 1 \right) $ time. We then 
                    call BFS twice which we know runs in $O\left( m + n \right) $ 
                    time. We then reach nested for loops that will only run
                    $O\left( m \right) $ times. All the operations within
                    the innermost for loop are $O\left( 1 \right) $, setting 
                    variables and accessing arrays to make logical comparisons.
                    Finally, the last few operations are all $O\left( 1 \right) $ 
                    time. The algorithm therefore runs in $O\left( m + n \right) $ 
                    time - limited by the BFS performed.

                    \textbf{Space Complexity}: The Space complexity is the space
                    it takes to store the graph or $O\left( m + n \right) $




            \end{itemize}

    \end{enumerate}



    \newpage
    \item(\textbf{Maximum clearance routes}) Every year, trucks get stuck on Storrow Drive because of the low bridges (sometimes called ``\href{https://duckduckgo.com/?q=storrowing&t=ffab&iar=images&iax=images&ia=images}{storrowing}''). Suppose you're running a trucking company in Boston with $n$ warehouses. Each section of road $e\in E$ will be annotated with the clearance of the lowest bridge on that portion of road $c_e$. Assume all of the roads are undirected, and that all of the clearances are unique.  We'll say that the \emph{clearance of a route} from $i$ to $j$ is the minimum clearance of all the road segments on the route (you can't drive a truck that's taller than the lowest bridge on the route). Next, we'll say that a route is a \emph{maximum-clearance route between $i$ and $j$} if it is the route between $i$ and $j$ with largest route clearance out of all possible routes from $i$ to $j$. Intuitively, the maximum-clearance route between two nodes tells you the tallest truck that you can drive between the two locations without getting stuck under a bridge, as well as which path to use.
    
    You want to find an algorithm that computes a tree for max-clearance routes from a given node $s$. After giving it some thought, you think it might be related to the idea of a maximum spanning tree with respect to the edge clearances.
    
    Note: the \textbf{maximum} spanning tree of a graph $G$ is just a tree that connects all vertices in $G$ (hence "spanning") while maximizing the sum of the edge weights in the tree. Cycles are not allowed, as we still want a \emph{tree}.
   
    
    \begin{enumerate}
        \item State the cut and  cycle property for \textit{maximum} spanning trees (MaxST). 
    
        \item Show that in any graph $G$ with unique edge clearances, the path given by using edges from the maximum spanning tree of G (with respect to the edge weights given by the clearances $c_e$) gives the maximum clearance route between any two warehouses. [\textit{Hint:} There are a couple of natural ways to do this. One way is to proceed by contradiction: suppose that the highest-clearance path is not part of the MaxST; show that together with edges in the MaxST it creates a cycle that contradicts the cycle property above.]
        
        \item Give a polynomial-time algorithm that takes a graph $G$ with distinct edge clearances and outputs the maximum spanning tree. \emph{Hint:} Modify any of the minimum-weight spanning tree algorithms from class.
    \end{enumerate}

    \begin{enumerate}
        \color{teal}
        \item \textbf{Cut and cycle property for maximum spanning trees}
            \begin{itemize}
                \item \textbf{Cut Property}

                    \color{red} do i need to prove anywhere? \color{teal}
                    The cut property for MaxSTs is the following: Assume that 
                    all edge costs are distinct. Let $S$ be any subset of nodes
                    that is neither empty nor equal to all of V, and let edge
                    $e = (v, w)$ be the maximum cost edge with one end in $S$ 
                    and the other in $V - S$. Then every maximum spanning tree 
                    contains the edge $e$. (Based on the property as defined 
                    in the textbook)

                \item \textbf{Cycle Property}

                    Assume that all edge costs are distinct. Let $C$ be any 
                    cycle in $G$, and let edge $e = (v, w)$ be the least 
                    expensive edge belonging to $C$. Then $e$ does not belong 
                    to any minimum spanning tree of $G$. (Based on the 
                    property as defined in the textbook)


            \end{itemize}
        \item \textbf{Show MaxST contains minimum clearance path between any two 
            nodes}
            
            Assume $M$ is the MaxST of graph $G(E, V)$ and there is an edge $e$ 
            that over which we can find  higher clearance path from $u \in V$ 
            to $v \in V$. If we add $e$ to $M$ we must create a cycle since 
            $M$ is a spanning tree and $e$ is an edge between two nodes in 
            $V$. To keep the highest clearance path in the cycle we must remove 
            the lowest weight edge in the newly created cycle per the cycle 
            property stated in part a. If $M$ was a MaxST before this, $e$ 
            would not have been added because it is a smaller weight path. 
            Therefore, by contradiction, the MaxST must comprise of the highest 
            clearance paths between all nodes. 

        \item \textbf{Polynomial-time algorithm that outputs the MaxST}

            \begin{itemize}
                \item \textbf{A precise description of the algorithm}
                    To write an algorithm that finds the maximum spanning tree
                    all we need to do is modify Kruskal's algorithm to 
                    initially sort the edges in descending order instead of the 
                    ascending order used for minimum spanning trees.

                \item \textbf{Proof of correctness}

                    Kruskal's relies on the cut property for adding each node. 
                    When sorting by descending edge weight instead of 
                    ascending, we add the greatest edge spanning two sets that
                    are disconnected. We know they are disconnected because we 
                    check that adding the new edge does not create a cycle. This
                    principle is consistent with the maximum spanning tree rule
                    as described in part a. 
                    \color{red} do i need to prove the part a? \color{teal}

                \item \textbf{Time and Space Complexity}
                    
                    \textbf{Time Complexity}: Since the only change in Kruskal's algorithm is the 
                    way in which the edges are sorted, which has the same time
                    complexity either way, the time complexity for the algorithm
                    stays at $m\log{n}$

                    \textbf{Space Complexity}: The Space complexity is the space
                    it takes to store the graph or $O\left( m + n \right) $


            \end{itemize}
            


    \end{enumerate}


\end{enumerate}
\end{document}
