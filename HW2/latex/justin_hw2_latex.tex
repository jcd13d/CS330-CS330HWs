
\documentclass[11pt]{article}

\usepackage{fullpage}
\usepackage{xcolor}
\usepackage{amsmath}
\usepackage{hyperref}
\usepackage[ruled,vlined,linesnumbered]{algorithm2e}


\pdfpagewidth 8.5in
\pdfpageheight 11in 



\oddsidemargin 0in
\evensidemargin 0in


\newtheorem{claim}{Claim}
\newtheorem{definition}{Definition}
\newtheorem{theorem}{Theorem}
\newtheorem{lemma}{Lemma}
\newtheorem{observation}{Observation}
\newtheorem{question}{Question}
\newtheorem{problem}{Problem}

\newcommand{\re}{{\mathbb{R}}}
\newcommand{\floor}[1]{\lfloor {#1} \rfloor}
\newcommand{\ceil}[1]{\lceil {#1} \rceil}
\newcommand{\paren}[1]{\left( {#1} \right)}


\begin{document}
	
\begin{center}
\Large{\textbf{CS 330, Fall 2020, Homework 2 \\
Due Wednesday, September
  16, 2020, 11:59 pm Eastern Time}}
\end{center}

\begin{center}
   Student: Justin DiEmmanuele \\
    Collaborators: Shilpen Patel, George Padavick, Matthew Gilgo
\end{center}



\section*{Homework Guidelines}

\paragraph{Collaboration policy} Collaboration on homework problems, with the exception of
programming assignments and reading quizzes, is permitted, but not encouraged.
If you
choose to collaborate on some problems, you are allowed to discuss
each problem with at most 5 other students currently enrolled in the
class.
Before working with others on a problem, you should think about it
yourself for at least 45 minutes. Finding answers to problems on the
Web or from other outside sources (these include anyone not enrolled
in the class) is strictly forbidden.

{\em You must write up each problem solution by yourself without
assistance, even if you collaborate with others to solve the
problem.} You must also identify your collaborators. If you did not
work with anyone, you should write "Collaborators: none." It is a
violation of this policy to submit a problem solution that you
cannot orally explain to an instructor or TA.

\paragraph{Solution guidelines} For problems that require you to provide an algorithm, you must give the following:
    \begin{enumerate}
\item  a precise description of the algorithm in English and, if helpful, pseudocode,
\item a proof of correctness,
\item an analysis of running time and space.
\end{enumerate}
You may use algorithms from class as subroutines. You may also use any facts that we proved in class.


You should be as clear and concise as possible in your write-up of
solutions. 

A simple, direct analysis is worth more points than a
convoluted one, both because it is simpler and less prone to error and
because it is easier to read and understand. Points might be
subtracted for illegible handwriting and for solutions that are too
long. Incorrect solutions will get from 0 to 30\% of the grade,
depending on how far they are from a working solution. Correct
solutions with possibly minor flaws will get 70 to 100\%, depending on
the flaws and clarity of the write up.

\newpage
\begin{enumerate}

  \item {\bf Stable Matching} (2-page limit, 15 points) 
\begin{enumerate}

\item ({\bf Unique Matching}) Give an algorithm that takes an instance of the stable
    matching problem as input and decides if there is {\em exactly
      one} stable matching for this instance. That is, the algorithm takes as input the preference lists for two groups of the same size, $n$ (say, hospitals and residents), and
    outputs either ``unique'', or ``more than one''. Your algorithm should run in polynomial time (for example, $O(n^2)$ or $O(n^{4.2})$; it shouldn't require exponential time).
    
    Notice that proving your algorithm correct requires proving three statements: 

    \begin{enumerate}
        \item The algorithm always terminates;
        \item If there is a unique stable matching, then the algorithm will output ``Unique'';
        \item If the algorithm outputs ``Unique'', then there really is only one stable matching. (This is the most challenging statement of the three)
    \end{enumerate}

    \emph{Hint}: It is helpful to read the section on ``Extensions'' at the
    end of KT Section 1.1; specifically, read and understand Claims 1.7 and 1.8.

    \color{teal}
    \paragraph*{Solution} The solution below relies on the Gale-Shapley algorithm as described in Section 1.1 of KT.
    
    \begin{algorithm}[H]
      \caption{UniqueMatching($male\_pref, female\_pref$)} 
      \KwIn{$ male\_pref$ is an array of real numbers shaped $\left( n, m \right) $ where $male\_pref[n, m]$ is the $n$th male's $m$th preference.}
      \KwIn{$female\_pref$ is the same as  $male\_pref$ but for the female participants.}

      $male\_best\_match, female\_worst\_match = GaleShapley\left( male\_pref, female\_pref \right) $

      $female\_best\_match, male\_worst\_match = GaleShapley\left(female\_pref , male\_pref \right) $

      \For{$i = 1$ to $n$}{
          \If{$male\_best\_match[i] \neq  male\_worst\_match[i]$}{
              \KwOut{Not Unique}
          }
          \Else{
              \# Do Nothing
        }
      }
      \KwOut{Unique}

    \end{algorithm}

    The requirements that need to be satisfied for this problem are discussed below:

    \begin{enumerate}
        \color{black}
        \item The algorithm always terminates;
        \color{teal}
            \begin{itemize}
                \item Gale-Shapley Portion (lines 1-2)
                    \begin{itemize}
                        \item This algorithm is proved to terminate in claim 1.3 in Section 1 of KT
                    \end{itemize}
                \item Array comparison (starting line 3)
                    \begin{itemize}
                        \item This section simply is composed of a for loop iterating from 1 to $n$. This will execute at most $n$ times before terminating.
                    \end{itemize}
            \end{itemize}
        \color{black}
        \item If there is a unique stable matching, then the algorithm will output ``Unique'';
        \color{teal}
            \begin{itemize}
                \item I assert that a unique solution exists if the most preferable G-S matching is the same as the worst matching.
                \item If this is not true the algorithm automatically outputs "Not Unique". Otherwise, "Unique" is output. This can be seen by studying line 3 on in algorithm 1.
                \item "Unique" can ONLY be output if the two compared arrays have every index matching.
            \end{itemize}
        \color{black}
        \item If the algorithm outputs ``Unique'', then there really is only one stable matching. (This is the most challenging statement of the three)
        \color{teal}
            \begin{itemize}
                \item We know from claim 1.6 in KT the G-S algorithm returns a set of matchings that are stable and from 1.8 this stable matching has the best possible matching for the proposing party. 
                \item Claim 1.7 tells us that every execution of the G-S algorithm with the same input results in the same output. (the algorithm is deterministic)
                \item Let $s$ represent the satisfaction level of the proposing side. This would be a measure of how high on the preference list each matching is for the proposing group as a whole.
                \item By definition any matching $m$ must satisfy the following inequality: 
                    \[
                    min\left( s \right)  \le m \le  max\left( s \right)  
                    .\] 
                \item If we run G-S with the male group both proposing and being proposed to we can find both $m_{1} = max\left( s \right)$ and $m_{2} = min\left( s \right) $.
                \item Then, all that is needed is to compare $m_{1}$ and $m_{2}$. If they are equal, the solution must be unique because there is no better and no worse solution from a satisfaction perspective from the inequality stated above.
            \end{itemize}

    \end{enumerate}

    \color{black}



    
\newpage
\item ({\bf Lower bounds}) Prove that it is possible for the Gale-Shapley algorithm to perform $\Omega(n^2)$
offers before termination. [\textit{Hint:} One way to do this is to describe, for each $n$, both a suitable input and
a sequence of  $\Omega(n^2)$ valid offers.]
    \color{teal}
\paragraph{Solution} 
    \begin{itemize}
        \item let $m$ and $s$ represent sets of men and women respectively. Both of length $n$.
        \item The men all have the same preference list:
            \[
            w_{1}, w_{2}, w_{3}, \ldots, w_{n}
            .\] 
        \item The women all have the same preference list:
            \[
            m_{n}, m_{n-1}, m_{n-2}, \ldots, m_{1}
            .\] 
        \item We can walk through proposals for n=3 to understand how many iterations are needed.
            \begin{itemize}
                \item $m_{1}$ proposes to $w_{1}$, she accepts
                \item $m_{2}$ proposes to $w_{1}$, she accepts. $m_{1}$ is free.
                \item $m_{3}$ proposes to $w_{1}$, she accepts. $m_{2}$ is free.
                \begin{itemize}
                    \item \textbf{This is the first stable match as $m_{3}$ and $w_{1}$ are each others first preference}
                \end{itemize}
                \item $m_{1}$ proposes to $w_{2}$, she accepts.
                \item $m_{2}$ proposes to $w_{2}$, she accepts. $m_{1}$ is free.
                \begin{itemize}
                    \item \textbf{This is the second stable match as $m_{2}$ only prefers $w_{1}$ to $w_{2}$ and $w_{2}$ only prefers $m_{3}$. $m_{3}$ and $w_{1}$ are currently in a stable matching.}
                \end{itemize}
                \item Finally, $m_{1}$ proposes to $w_{3}$, she accepts.
                \begin{itemize}
                    \item \textbf{This is the final match}
                \end{itemize}
            \end{itemize}
        \item This shows the characteristics of this preference list case. The resulting number of iterations is: \[
                n + (n-1) + (n-2) + \ldots + 2 + 1 = n\frac{n-1}{2}
        .\] 
        \begin{itemize}
            \item Each stable matching must iterate through every available man because the last man is the one who will find a stable match. 
            \item Each stable match takes 1 less iteration to find since the previous iteration results in a stable match.
        \end{itemize}
    \item In the expanded version of the equation representing the number of iterations it is clear G-S is $\Theta\left( n^2 \right) = \Omega\left( n^2 \right) $\[
            n\frac{n-1}{2} = \frac{n^2}{2}-\frac{n}{2}-\frac{1}{2} = f\left( n \right) 
    .\] \[
            g\left( n \right) = n^2
    .\] \[
    \lim_{n \to \infty} \frac{f\left( n \right) }{g\left( n \right) }= \frac{1}{2}
    .\]
    \end{itemize}
    \color{black}

\end{enumerate}


\newpage
\item {\bf Asymptotics Review} (1-page limit, 10 points)

Rank the following functions by their asymptotic growth rate (big-O relationships). 

List functions on separate lines, where higher lines correspond to asymptotically larger functions (so $f(n)$ goes above $g(n)$ if $g(n)=O(f(n))$ and $f(n)\neq O(g(n))$). If two functions are asymptotically equivalent (that is, $f(n)=O(g(n))$ and $g(n)=O(f(n))$), then they should go on the same line.


\[
\renewcommand{\arraystretch}{2}
\begin{array}{c|c|c|c|c}
 \makebox[1in]{\ensuremath{n^{100}}}& 
 n^3 - 5n^2 -10 n + 12 & 
  \makebox[1in]{\ensuremath{n^{2} }}& 
  \makebox[1in]{\ensuremath{\sqrt{n} }} & 
  \makebox[1in]{\ensuremath{2^n }}
\\ \hline 
n^3 & \log_2 n & \log_3 n  & \sum_{i=1}^n i & 2.1^n\\ \hline 
10 +\frac{1}{n} & n\log_2{n} & n\cdot 2^n & \binom{n}{2} - n + 2 &\log_2(n!)\\
\end{array}
\renewcommand{\arraystretch}{1}
\]


\emph{Hint}: You may want to review Big-Oh rules and logarithm properties.



\[
%\tiny
%\renewcommand{\arraystretch}{2}
%\begin{array}{c|c|c|c|c}
    %\makebox[1in]{\ensuremath{n^{100} = O\left( n^{100} \right) }}& 
    %n^3 - 5n^2 -10 n + 12 = O\left( n^{3} \right) & 
    %\makebox[1in]{\ensuremath{n^{2} = O\left( n^{2} \right) }}& 
    %\makebox[1in]{\ensuremath{\sqrt{n} = O\left( \sqrt{n}  \right) }} & 
    %\makebox[1in]{\ensuremath{2^n = O\left( 2^n \right) }}
%\\ \hline 
    %n^3 = O\left( n^3 \right) & \log_2 n = O\left( \log{n} \right) & \log_3 n = O\left( \log{n} \right)   & \sum_{i=1}^n i = O\left( n^2 \right)  & 2.1^n = O\left( 2.1^n \right) \\ \hline 
    %10 +\frac{1}{n} = O\left( n^{-1} \right) & n\log_2{n} = O\left( n\log{n} \right) & n\cdot 2^n = O\left( n \cdot 2^{n} \right) & \binom{n}{2} - n + 2 = O\left( n^2 \right)  &\log_2(n!) = O\left( n\log{n} \right) \\
%\end{array}
%\renewcommand{\arraystretch}{1}
\]



\color{teal}
\paragraph{Solution} The functions are shown below sorted by asymptotic growth rate.

\begin{table}[htpb]
\color{teal}
    \centering
    \caption{Asymptotic Sorting}
    \begin{tabular}{c|c|c}
    $n\cdot2^{n}$ &  &  \\ \hline
    $2.1^{n}$  &  &  \\ \hline
    $2^{n}$ &  &  \\ \hline
    $n^{100}$ &  &  \\ \hline
    $n^3 - 5n^2 - 10n + 12$ & $n^3$ &  \\ \hline
    $\sum_{n=i}^{n} i$ & $n^2$ & ${n \choose 2}-n +2$ \\ \hline
    $n\log_2{n}$ & $\log_2{n!}$ &  \\ \hline
     $\sqrt{n} $ &  &  \\ \hline
     $\log_2{n}$ & $\log_3{n}$ &  \\ \hline
    $10 + \frac{1}{n}$ &  &  \\ \hline
    \end{tabular}
\end{table}

\end{enumerate}
\end{document}
