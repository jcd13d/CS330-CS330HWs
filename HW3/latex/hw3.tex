\documentclass[letterpaper,11pt]{article}

\usepackage{fullpage,amsmath,amsfonts,latexsym,xcolor,clrscode3e}
\usepackage{graphicx}
\usepackage{amsthm}
\usepackage{hyperref}
\usepackage{fullpage}
\usepackage[ruled,vlined,linesnumbered]{algorithm2e}

\newcommand{\re}{{\mathbb{R}}}
\newcommand\numberthis{\addtocounter{equation}{1}\tag{\theequation}}
\newcommand{\floor}[1]{\lfloor {#1} \rfloor}
\newcommand{\ceil}[1]{\lceil {#1} \rceil}
\newcommand{\paren}[1]{\left( {#1} \right)}
\newenvironment{solution}{\color{black} }{}

\newcommand{\nats}{\mathbb{N}}

\newcommand{\comment}[1]{$\rhd$\ {\small\sf #1}}

\newtheorem{theorem}{Theorem}
\newtheorem{claim}[theorem]{Claim}
\newtheorem{lemma}[theorem]{Lemma}
\newtheorem{problem}{Problem}


\begin{document}
{\noindent\large
{\em Introduction to Analysis of Algorithms} \hfill \today\\
Boston University \hfill CS 330\\
Professor  Adam Smith, Dora Erdos \hfill Fall 2020\\}
\vspace{1pt}
\hrulefill\vspace{3mm}
\begin{center}
{\LARGE\bf Homework 3}

\begin{center}
    \color{teal}
   Student: Justin DiEmmanuele \\
    Collaborators: Shilpen Patel, George Padavick, Matthew Gilgo
\end{center}


\end{center}

%Solutions to part 1 go here. It might be helpful to copy the problem statement. Use {\color{blue} blue text if you are reproducing the problem statement} and black text for solutions. 

\begin{problem}
 Alternating Paths (10 pts.)
\end{problem}

Let $G(V,E, s, c)$ be an undirected graph with $s$ as a source node.  Each of its edges $e \in E$ is either blue or red, and $c_e$ denotes the color of edge $e$. Let all edges adjacent to $s$ be blue. 

A path between nodes $u$ and $v$ in graph $G$, called $P_{u \rightarrow v}$, is a \emph{shortest} path if there is no other path between $u$ and $v$ with fewer edges. $P_{u \rightarrow v}$ is called \emph{simple} if it never visits the same edge or node twice.  A simple path $P$ in $G$ is called \emph{alternating} if the edge colors in the path alternate between red and blue.  Find an algorithm that takes as input $G(V,E, s, c)$ and computes whether there is a \emph{shortest alternating path} from $s$ to every other node. If that is the case, then the output should be the alternating path to each node, which should be represented as ordered lists of edges.  Otherwise, return ``No". 

\begin{enumerate}
    \color{teal}
    \item \textbf{A precise description of the algorithm in English and, if helpful, pseudocode}

        To achieve the desired results, some changes need to be made to the BFS 
        algorithm as described in section 3.3 of \textit{Algorithm Design} 
        (Kleinberg and Tardos). We need to handle the extra constraint that a 
        path must be alternating as described in the problem statement. To do
        this, we need to track the colors of the previous edge  as well as the
        color of the edge leading to the children we are investigating. Then, 
        in addition to checking if a node has been discovered yet, we will also 
        check if the previous edge color is different from the edge color 
        leading to the child. This effectively only discovers a node if it 
        satisfies the alternating requirement.

        We assume that we take an adjacency list with tuples 
        containing adjacent node and color. In all our data structures, we 
        maintain the information on the color of the edge. When we chose to 
        investigate a node for its adjacent nodes we pull that nodes previous
        color. As we visit each adjacent node's edge color to the previous one
        and only add the node to the tree and investigation queue if its edge's 
        color is not equal to the edge we are investigating from.
        
        The output of the algorithm is a tree stemming from the source root to 
        the leaves in which the level corresponds to the distance from that 
        source. To get the alternating path to each node, we must traverse the 
        tree from each node to the source (if the path exists).

        \begin{algorithm}[H]
            \label{alg:test}
            \caption{GetPaths($G\left( V, E, s, c \right)$)} 
            \KwIn{$ G$ is an undirected graph as described in the problem 
            statement}
            \KwIn{$s$ is the source node of graph $G$ }
            $bfs\_tree \gets BfsAlternatingColor\left( G \right) $ \;
            $n \gets length(A)$ \;
            $paths =  Empty Array Length\_n$ \;
            \For{$i\gets 1$ to $n$} 
            {
                $path = [] $ \# empty list
                $parent = i$ \;
                \While{$parent \neq s$}{
                    \If{$parent$ in $bfs\_tree$}{
                        $child = parent$ \;
                        $parent = bfs\_tree[child].get\_parent()$ \;
                        $path.append\left( \left( parent, child \right)  \right) $\;
                    }
                    \Else{
                        \KwOut{$EmptyList[]$}
                    }
                }
                $path.reverse()$ \;
                $paths[i] = path$ \;
            }
            \KwOut{$paths$}
        \end{algorithm}


        In this question we are interested in if there is a path that is both 
        the shortest path and alternating between the source and every other
        node. Since out modified version of BFS will return all the alternating 
        paths and the original version will return the shortest paths, we can 
        simply compare the length of the output for each node. This is the 
        length of each array in $paths$ as returned in Algorithm 1. We run 
        Algorithm 1 using the modified alternating BFS and original BFS then 
        run:

        \begin{algorithm}[H]
            \caption{ComparePaths($path\_alternating, path\_normal$)} 
            \KwIn{$path\_alternating$ is the result of $GetPaths\left( \  \right) $ when run on the modified BFS - an array length $n$ with each index $i$ containing an array of tuples that represent the edges that make up the path to node $i$}
            \KwIn{$path\_normal$ is the same as $path\_alternating$ but run using the original BFS}
            \For{$ i=1 $ to $n$ }{
                \If{$paths\_alternating[i] != paths\_normal[i]$}{
                    \KwOut{"NO"}
                }
            }
            \KwOut{$paths\_alternating$ }
        \end{algorithm}

        This gives the desired result - the alternating path to each node from 
        the source given that it is also the shortest path or "NO".

    \item \textbf{A proof of correctness}
        \begin{itemize}
            \item Modified Breadth-First-Search 

                The modified BFS algorithm used in algorithm 1 follows the same 
                claim made in KT - claim 3.3. In the original version of the 
                algorithm, any new node that is found is added to the graph as 
                long as it has not been discovered. For the special case, we 
                just add a further reaching constraint that the node must not 
                only have not been discovered, it must have a different color 
                then the edge leading to its parent.

                Since the only changes to BFS are simple statements that are
                guaranteed to terminate we can be confident the modified 
                version will terminate.

            \item BFS Tree Traversal 
                
                Given that the returned BFS tree is correct, we must prove
                algorithm 1 correctly traverses that tree to return the set
                of edges leading to each destination node. In a BFS tree, each
                node appears once and each node has only 1 parent. This is true 
                because of the "if discovered" check before a node is added to
                the tree and the fact that there is no option to give a node 
                two parents per the definition of a tree.

                Because of these facts, we can simply start at each destination 
                and iteratively call for its parent until we reach the source 
                or root node. If a node does not exist in the tree it has no 
                valid alternating path per claim 3.3.

                Since every node of the tree has just one parent per the 
                definition of a tree, and we know BFS returns a tree, we can be 
                sure the program will terminate.

            \item Comparison
                
                The comparison simply goes index by index of the array and 
                compares the alternating version and original BFS versions. If 
                they are different the algorithm immediately halts and returns
                "No". Otherwise, the program runs through the rest of the nodes'
                paths then outputs the shortest alternating paths for the graph.

                The comparison runs a finite $n$ times. We can be sure the
                program will terminate.

        \end{itemize}

    \item \textbf{An analysis of running time and space}

        \begin{itemize}
            \item Running Time

                \begin{itemize}
                    \item Modified BFS Running Time

                        The modified BFS will still run in $O\left( m+n \right) $ 
                        time per claim 3.11 in KT. The only changes to the 
                        algorithm are $O\left( 1 \right) $ - look-ups for 
                        edge color and logical comparisons to check if colors
                        are alternating. 

                    \item Tree Traversal Running Time

                        The tree traversal shown in algorithm 1 will run in 
                        $O\left( n \right) $ time. Where $n$ is the number of
                        nodes in the tree. At most a path can 
                        be length $n$ since a path is the number of edges 
                        separating two nodes. The tree traversal will be applied
                        to at most $n$ nodes. 

                    \item Comparison Running Time

                        The comparison will need to iterate through the shortest 
                        paths for $n$ nodes. This will smaller than iterating 
                        through the graph adjacency list which has a run time of
                        $O\left( n + m \right) $. The shortest paths contain the 
                        same number of nodes as the adjacency list but fewer edges
                        since not all are needed to describe the shortest path.

                    \item This makes the running time $O\left( n + m \right) $ 
                        limited by the BFS search algorithm.

                \end{itemize}


            \item Space Complexity

                The largest structure the algorithm will need to handle is the 
                adjacency-list passed representing the graph. All data 
                other data structures will take less space than this as they 
                do not need to store the complete information of the graph. This
                makes the space complexity $O\left( n + m \right) $ per the 
                lecture 4 slides.

        \end{itemize}

\end{enumerate}

\newpage
\newcommand{\Null}{\text{\tt null}}
\begin{problem} Binary Trees (15 pts.)
  \end{problem}

Let us represent a binary tree in the computer as follows.
 Each node of the tree $ T$ is a record,  $r $ is the root node.
 In a node $v$, if $v.left \neq \Null$ then it is the left child of $v$;
similarly for $v.right$.



  
  \paragraph{a.} (5 points) Write a recursive program (in pseudocode) to compute the number of leaves.
  
  \begin{enumerate}
      \color{teal}
      \item A precise description of the algorithm in english and, if helpful, 
          pseudocode

        \begin{algorithm}[H]
            \caption{RecursiveLeafCount($r$)} 
            \KwIn{$r$ is the root node of a binary tree}
            \If{$r$ is Null}{
                \KwOut{0}
            }
            \uElseIf{$r.left\left(  \right)$ and $r.right\left(  \right)$ are Null}{
                \KwOut{1}
            }
            \Else{
                \KwOut{$RecursiveLeafCount\left( r.left\left(  \right)  \right) + RecursiveLeafCount\left( r.right\left(  \right)  \right) $}
            }
        \end{algorithm}


      \item A proof of correctness

          There are a few cases we can investigate and test against this 
          algorithm to understand if it returns the desired result.
          \begin{itemize}
              \item Root node $r$ has no children.

                  In this case, the algorithm will trigger the else if
                  statement on line 2 since both children are null. The output
                  will correctly be 1 leaf.

              \item Root node $r$ is $null$ or does not exist.

                  The first if statement will be triggered and zero is correctly 
                  returned.

              \item A child exists.

                  The final else statement is executed and the function is called
                  on both children.

          \end{itemize}

          For each recursive call one of these cases will apply and they will 
          still hold true. We know that this algorithm will terminate because 
          the algorithm can only move down in layers of the tree. As we move 
          down in layers, the portion of the tree that has not been visited
          gets smaller. The function will be applied recursively until the 
          entire tree has been visited.

      \item An analysis of running time and space

          The operations in the $RecursiveLeafCount\left( \  \right) $ are all 
          $O\left( 1 \right) $. They are simply accessing variables and running
          logical comparisons. The question then becomes, how many times will 
          the algorithm be run on a tree of $n$ nodes? The algorithm only makes
          calls down the levels of the tree to every child. The algorithm 
          therefore will be $O\left( n \right) $.
  \end{enumerate}

\paragraph{b.} (10 points) {\bf Kelinberg and Tardos, Chapter 3, Problem 5}

Show by induction that in any binary tree the number of nodes with two children is exactly one less than the number of leaves.

    \color{teal}
        
        \begin{itemize}
            \item Base Case

                Take the simplest case - a tree with one root node with two 
                children. These children have no children of their own so they 
                are leaves. Let $N$ represent the number of nodes with two leaves
                and $L$ represent the number of leaves. We want to prove $L - 1 = N$ 
                \[
                L - 1 = N
                .\] 
                \[
                2 - 1 = 1
                .\] 

            \item Induction Step
                
                Take a node with two leaves at the highest level of the tree.
                The tree has $N_1$ nodes and $L_1$ leaves. If we remove both leaves, 
                the tree loses two leaves, but the parent node becomes a leaf 
                itself. Net, the tree loses one leaf. The tree also loses a node
                since the node in question becomes a leaf. This leaves us with 
                $N_1^{\prime} = N_1 - 1$ nodes and $L^{\prime}_1 = L_1 - 1$ leaves.
                
                If we plug this into 
                the equation\ldots
                \[
                L^{\prime}_1 - 1 = N^{\prime}_1
                .\] 
                \[
                    \left( L_1 - 1 \right)  - 1 = \left( N_1 - 1 \right) 
                .\] 
                \[
                L_1 - 2 = N_1 - 1
                .\] 
                \[
                L_1 - 1 = N_1
                .\] 

                We end up with the original statement: $L - 1 = N$ 

        \end{itemize}


\end{document}
